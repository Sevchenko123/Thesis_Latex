\chapter{Summary of Findings and Future Work}

Online social networks will only gain popularity in the future for connecting with friends, families and dear ones. This thesis presented a data driven approach to shed light on how users of leading emotional support service choose to connect with each other, how they behave, utilize the platform, what design choices and user behavior is responsible for high user engagement, discovered the frequency, kinds, and characteristics of cyberbullying on this system and investigated the effect of ‘blocking’ a cyberbully, which is the typical method of thwarting their behavior on the social service. We summarize our findings as follows:

1. As the goal of the platform is to provide emotional support and empathy to those who need it, the process of registering as a member on the website could actually lead to more positive health outcomes as compared to not registering and using the platform merely as a guest.

%\newpage
2. The gamification mechanisms integrated in the website are greatly responsible for user engagement. 

3. The structural characteristics of network of this platform leads to some unique and useful insights. The giant connected component of the bipartite network include almost all the users of the platform, which tells us that only few users choose to exclusively search for and speak with each other. The degree distribution of listener projection network follows a power law distribution but same may not be said for the degree distribution of member projection network. This suggest that member may tend to develop deep and strong relationships with some of the listeners rather than having an "exploratory" behavior where they are trying to connect with as many listeners as possible. The clustering coefficients of the member and listener projection appears to be normally distributed as is seen in many co-occurrence networks. A small percentage of members and listeners exhibit perfect clustering coefficients which is unique to this platform.
\footnote{Portions of this thesis were previously written and published by the author in~\cite{doran2015stay}.}

4. The menace of cyberbullying exist on this platform as well. Types of cyberbullying on this platform can be thematically described as sexual harassment, rude behavior, and as trying to acquire personal information.

%\newpage
5. The worst bullies that are blocked multiple times are also extremely active on the platform and are seasoned users which suggest that such worst bullies initiate a comfortable repertoire with a victim before attacking them personally. It therefore makes it difficult to conclude whether heavy tail is a natural phenomenon in an emotional support system, or if it is because of behavior of such worst bullies and some exceptional normal users. 

6. The act of ‘blocking’ has a strong effect on bullies, with a larger average number of conversations and larger standard deviation of conversations per day prior to a block event as compared to when the block(s) had been issued against them. This suggests that block could act as a deterrent to cyberbulling for most of the bullies. 


These findings have many implications about how people use crowdsourced emotional support systems. As mentioned before, the member projection network doesn't quite follow power law distribution which suggest that their behavior may not be exploratory in nature where the objective is to connect with many listeners possible but on the contrary it suggests that they may have thoughtful conversations with only few listeners and prefer to connect with them which suggest that members tend to choose to have conversations with their listeners very carefully, maybe only after reviewing their profile, ratings, reviews and overall experience. This may be due to the fact that they are not comfortable to open up about their problems to any listener and therefore prefers to connect with very experienced listener only. So selection of a listener for having a conversation looks highly competitive. This may also suggest that the members on the website are very serious regarding their emotional problems and to find solutions for it. This looks a positive behavior that the members are extra cautious while connecting with a listener and so in a way are more likely of not facing cyberbullying by a nefarious listener as compared to an unregistered guest who may opt to connect to whichever listener available for conversation. On the other hand, the degree distribution of listener projection network follows a power law distribution suggesting that some listeners are willing to help and support to as many members as possible. Also the fact, that listeners are able to support for number of different emotional problems again reaffirms their willingness to help multiple users facing multiple emotional problems. Engagement analysis also demonstrates the importance of gaming mechanisms for tracking user progress, which may serve as additional motivation for the users to help each other. Cyberbullying analysis also explored the problems of cyberbullying on this platform and investigated the effectiveness of standard measures adopted to thwart cyberbullying. The findings show some commonalities between users using other online social networks and users using 7cot such as normally distributed clustering coefficients of the users, while at the same time it also revealed some unique insights which only the users of 7cot tend to follow. For eg. small percent of users having perfect clustering coefficients. 

\newpage
Future work will include studying how cyberbullying grows or spreads, for example if a person have been cyberbullied, how likely is it that he/she will cyberbully somebody else, or in other words does being a victim of cyberbullying encourages the victim to cyberbully someone else ? Does cyberbullying spread like a contagious disease or an epidemic on an online platform and if so how can it be contained ? We would study the networks of the bullies and that of the victims who got cyberbullied and will analyze whether the structural properties of these networks show significant distinctions which could lead to some new insights about the behavior of the cyberbullies as compared to normal users. We will also explore other machine learning models to improve engagement analysis.

