\chapter{Related Research}

Studies in the past have tried to understand the users and their behaviors on online emotional support systems. Maloney-Krichmar et al. investigated the dynamics of group interactions among an online self-help group for knee injuries ~\cite {maloney2005multilevel}. Barak et al. established a positive relationship between the amount of activity of adolescents in an online support group and the emotional relief they felt, underscoring the importance of building online systems that facilitate user interactions ~\cite {barak2007emotional}. Ploderer et al. delved into the discussion topics on a Facebook group of people trying to quit smoking, and found that most supportive responses come from those who just began trying to quit, rather than long-term quitters ~\cite {ploderer2013patterns}. Yuen et al. highlighted how remote assessment, treatment and consultation which are provided via internet through self help websites, and videoconferencing have great potential to increase access to high quality psychological services. They also discussed clinical, ethical, logistical challenges involving security, competence, usability and technical difficulties on such platforms ~\cite {yuen2012challenges}. Wang et al. analyzed the relationship of emotional and informational support a user is exposed to on online support groups to their commitment in online health support groups ~\cite{wang2012stay}. Zhang et al. studied a facebook diabetes group and concluded that participating in such online groups have a lot of advantages for the users to share important information regarding diabetes irrespective of where they came from, their differences in languages and diversity ~\cite{zhang2013facebook}. Valerie et al. emphasized that online social networks play an important role in supporting different types of decision making, as they provide their participants various forms of support, ranging from the instrumental to emotional and informational ~\cite{sadovykh2015decision}. Saha et al. demonstrated how online social networks are easy and accessible communication platforms particularly in the context of users of autism to help, share and connect with other users having autism or with their families, caregivers etc. which helps them considerably by extracting useful information and at the same time also provide them social support ~\cite{saha2015demonstrating}. Past work have also studied characteristics of online social networks, for example Mislove et al. studied characteristics of many online social network graphs such as flickr, youtube, LiveJournal, orkut at a large scale to understand the properties of these networks so as to improve the design of such systems and designing new applications which could promote high user engagement ~\cite{mislove2007measurement}. Han et al. compared structural properties of Weibo and Twitter networks to understand differences and nuances of how the users use these online social systems differently ~\cite{han2015weibo}. 

A lot of work have also been done on studying cyberbullying on online platforms. For instance, An October 2014 Pew research survey offers the best evidence that cyberbullying is a major phenomenon that impacts Internet and social media users. Academic studies have also demonstrated the negative factors associated with cyberbullies that attack U.S. teenagers, and unearthed the fact that bullying is intrinsic to users rather than to a particular platform. In other words, no matter the online social system, cyberbullying should be expected to occur. Previous studies also demonstrate the ill effects of cyberbullying to its victims which could be as grave as suicide ~\cite{hosseinmardi2015detection}. Past works have also tried to understand how users utilize online social networks and social media to manage emotional and personal problems. For example, Newman, et al. showed that people are cautious while sharing health related information online on social networks like Facebook. They also demonstrated that users are hesitant to share certain types of information, especially personal or information that may be used as fodder for cyberbullies, to protect themselves and to manage their online impression ~\cite{newman2011s}. 

Thus past studies have covered various aspects of online social networks such as understanding structural properties of networks to better understand the use of such systems, thereby suggesting recommendations for improving such systems, demonstrated advantages of using online social communities to share information about wide range of problems such as emotional or medical problems. On the other hand there also has been lot of work 
regarding the menace of cyberbullying on online systems, the adverse effects of cyberbullying on the victims and the challenges associated with it. 

All the above mentioned aspects regarding online social networks have been explored individually in the previous studies. Also very little work has been done studying how the platforms providing online emotional support are being utilized by the users and whether such platforms are advantageous and popular. This thesis cohesively covers all the above mentioned aspects such as studying structural properties and characteristics of the online social networks, understanding user engagement, user behavior and finally understanding the problems and challenges of cyberbullying from the context of online social network providing emotional support. 
%Samir, continue expanding related work. References to just 5 other pieces are not enough, this should ideally be 10-15 papers. We then need a concluding paragraph that talks about how this thesis is different from the work you have presented. 
