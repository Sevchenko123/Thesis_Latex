\newpage


\clearpage\thispagestyle{empty}
\thispagestyle{plain}
\pagenumbering{roman}
\setcounter{page}{3}


\begin{center}
{\normalfont\Large\bfseries Abstract}
\end{center}

\singlespacing
{\noindent
Samir, Yelne.  %% last, first name, upper-lower case mix
M.S.C.E., Department of Computer Science and Engineering,
Wright State University,
2016.                           %% this year
Measures of User Interactions, Conversations, and Attacks in a Crowdsourced Platform Offering Emotional Support
.} %% title

\par\vskip 1.5cm

\doublespacing
%%%%%%%%%%%%%%%%%%%%%%%%%%%%%%%%%%%%%%%%%%%%%%%%%%%%%%%%%%%%%%%%%%%%%%%%%%%%%%%%%%%%Type or paste your abstract of your thesis here

Online social systems have emerged as a popular medium for people in society to communicate with each other. Among the most important reasons why people communicate is to share emotional problems, but most online social systems are uncomfortable or unsafe spaces for this purpose. This has led to the rise of online emotional support systems, where users needing to speak to someone can anonymously connect to a crowd of trained listeners for a one-on-one conversation. To better understand who, how and when users utilize these systems, and to evaluate their safety, this thesis offers a comprehensive examination of the characteristics of users and their interactions from a massive, leading emotional support platform. From a big data set of millions of conversations across hundreds of thousands of users, the study employs statistical measurement techniques and predictive analytics to shed light about the ways these platforms are utilized, and the extent to which users behave in un-wanting ways. The analysis leads to recommendations on promoting positive system utilization and an understanding of the effectiveness of protections in place to thwart emotional attacks. This work is likely the first to measure the activities and interactions in an online social system for emotional support. 


\restoregeometry
